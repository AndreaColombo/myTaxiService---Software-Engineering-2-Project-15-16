\documentclass[a4paper, 11pt, titlepage]{article}
\usepackage{graphicx}
\usepackage{titlesec}
\usepackage{hyperref}
\usepackage[export]{adjustbox}
\graphicspath{{Images/}}
\setcounter{secnumdepth}{4}
\hypersetup{
	colorlinks=true,
	linkcolor=black,
	filecolor=magenta,      
	urlcolor=cyan,
}
%opening
\title{
	\begin{figure}[h]
		\centering
		\includegraphics{polimi_logo}
	\end{figure}
	Politecnico di Milano, A.A. 2015/2016 
	\newline\newline 
	Software Engineering 2: \textbf{C}ode \textbf{I}nspection
	}
\author{Belotti Nicola 793419\\ Chioso Emanuele 791621\\ Colombo Andrea 853381}


\begin{document}
	
	\maketitle
	\tableofcontents
	\newpage
	
\section{Introduction}

\subsection{Purpose}

\subsection{Scope}

\subsection{Definitions, Acronyms, Abbreviations}

\subsection{Reference Documents}

\subsection{Document Structure}

\section{Assigned Methods}
\subsection{First Method: getConstructor} \\ \\

	/** Returns a wrapped constructor element for the specified argument types 
	 * in the class with the specified name.  If the specified class name is 
	 * a persistence-capable key class name which corresponds to a bean 
	 * with an unknown primary key class a dummy constructor will also be 
	 * returned.  Types are specified as type names for primitive type 
	 * such as int, float or as fully qualified class names.
	 * @param className the name of the class which contains the constructor 
	 * to be checked
	 * @param argTypeNames the fully qualified names of the argument types
	 * @return the constructor element
	 * @see #getClass
	 */
	 
\begin{lstlisting}	public Object getConstructor (final String className, String[] argTypeNames)
	{
		Object returnObject = null;

		if ((NameMapper.PRIMARY_KEY_FIELD == 
			getPersistenceKeyClassType(className)) && 
			Arrays.equals(argTypeNames, NO_ARGS))
		{
			returnObject = new MemberWrapper(className, null, Modifier.PUBLIC, 
				(Class)getClass(className));
		}

		if (returnObject == null)
		{
			returnObject = super.getConstructor(className, argTypeNames);

			if (returnObject instanceof Constructor)		// wrap it
				returnObject = new MemberWrapper((Constructor)returnObject);
		}

		return returnObject;
	}
	
\end{lstlisting}
	
	
\newpage
\subsection{Second Method: getMethod} \\ \\


/** Returns a wrapped method element for the specified method name and \\
	 * argument types in the class with the specified name.  If the \\
	 * specified className represents a persistence-capable class and \\
	 * the requested methodName is readObject or writeObject, a dummy \\
	 * method will be returned.  Similarly, if the specified class name is 
	 * a persistence-capable key class name which corresponds to a bean 
	 * with an unknown primary key class or a primary key field (in both 
	 * cases there is no user defined primary key class) and the requested 
	 * method is equals or hashCode, a dummy method will also be returned.  
	 * Types are specified as type names for primitive type such as int, 
	 * float or as fully qualified class names.  Note, the method does not 
	 * return inherited methods.
	 * @param className the name of the class which contains the method 
	 * to be checked
	 * @param methodName the name of the method to be checked
	 * @param argTypeNames the fully qualified names of the argument types
	 * @return the method element
	 * @see #getClass
	 */
\begin{lstlisting}
	
	public Object getMethod (final String className, final String methodName,
		String[] argTypeNames)
	{
		int keyClassType = getPersistenceKeyClassType(className);
		Object returnObject = null;

		if (isPCClassName(className))
		{
			if ((methodName.equals("readObject") && 	// NOI18N
                    Arrays.equals(argTypeNames, getReadObjectArgs())) ||
				(methodName.equals("writeObject") && 	// NOI18N
                        Arrays.equals(argTypeNames, getWriteObjectArgs())))
			{
				returnObject = new MemberWrapper(methodName, 
					Void.TYPE, Modifier.PRIVATE, (Class)getClass(className));
			}
		}
		if ((NameMapper.UNKNOWN_KEY_CLASS == keyClassType) || 
			(NameMapper.PRIMARY_KEY_FIELD == keyClassType))
		{
			if (methodName.equals("equals") && 	// NOI18N
                    Arrays.equals(argTypeNames, getEqualsArgs()))
			{
				returnObject = new MemberWrapper(methodName, 
					Boolean.TYPE, Modifier.PUBLIC, (Class)getClass(className));
			}
			else if (methodName.equals("hashCode") && 	// NOI18N
                    Arrays.equals(argTypeNames, NO_ARGS))
			{
				returnObject = new MemberWrapper(methodName, 
					Integer.TYPE, Modifier.PUBLIC, (Class)getClass(className));
			}
		}

		if (returnObject == null)
		{
			returnObject = super.getMethod(className, methodName, argTypeNames);

			if (returnObject instanceof Method)		// wrap it
				returnObject = new MemberWrapper((Method)returnObject);
		}

		return returnObject;
	}	
\end{lstlisting}

\newpage
\subsection{Third Method: getFields} \\ \\

/** Returns a list of names of all the declared field elements in the 
	 * class with the specified name.  If the specified className represents 
	 * a persistence-capable class, the list of field names from the  
	 * corresponding ejb is returned (even if there is a Class object 
	 * available for the persistence-capable).
	 * @param className the fully qualified name of the class to be checked 
	 * @return the names of the field elements for the specified class
	 */
\begin{lstlisting}
	
	public List getFields (final String className)
	{
		final EjbCMPEntityDescriptor descriptor = getCMPDescriptor(className);
		String testClass = className;

		if (descriptor != null)		// need to get names of ejb fields
		{
			Iterator iterator = descriptor.getFieldDescriptors().iterator();
			List returnList = new ArrayList();

			while (iterator.hasNext())
				returnList.add(((FieldDescriptor)iterator.next()).getName());

			return returnList;
		}
		else
		{
			NameMapper nameMapper = getNameMapper();
			String ejbName = 
				nameMapper.getEjbNameForPersistenceKeyClass(className);

			switch (getPersistenceKeyClassType(className))
			{
				// find the field names we need in the corresponding 
				// ejb key class
				case NameMapper.USER_DEFINED_KEY_CLASS:
					testClass = nameMapper.getKeyClassForEjbName(ejbName);
					break;
				// find the field name we need in the abstract bean 
				case NameMapper.PRIMARY_KEY_FIELD:
					return Arrays.asList(new String[]{
						getCMPDescriptor(ejbName).
						getPrimaryKeyFieldDesc().getName()});
				// find the field name we need in the persistence capable 
				case NameMapper.UNKNOWN_KEY_CLASS:
					String pcClassName = 
						nameMapper.getPersistenceClassForEjbName(ejbName);
					PersistenceFieldElement[] fields = 
						getPersistenceClass(pcClassName).getFields();
					int i, count = ((fields != null) ? fields.length : 0);

					for (i = 0; i < count; i++)
					{
						PersistenceFieldElement pfe = fields[i];

						if (pfe.isKey())
							return Arrays.asList(new String[]{pfe.getName()});
					}
					break;
			}
		}

		return super.getFields(testClass);
	}
\end{lstlisting}


\newpage
\subsection{Fourth Method: getField} \\ \\
	/** Returns a wrapped field element for the specified fieldName in the 
	 * class with the specified className.  If the specified className 
	 * represents a persistence-capable class, a field representing the 
	 * field in the abstract bean class for the corresponding ejb is always 
	 * returned (even if there is a Field object available for the 
	 * persistence-capable).  If there is an ejb name and an abstract bean
	 * class with the same name, the abstract bean class which is associated
	 * with the ejb will be used, not the abstract bean class which  
	 * corresponds to the supplied name (directly).
	 * @param className the fully qualified name of the class which contains
	 * the field to be checked
	 * @param fieldName the name of the field to be checked
	 * @return the wrapped field element for the specified fieldName
	 */
	 
\begins{lstlisting}
\begin{code}
public Object getField (final String className, String fieldName)
	{
		String testClass = className;
		Object returnObject = null;

		if (className != null)
		{
			NameMapper nameMapper = getNameMapper();
			boolean isPCClass = isPCClassName(className);
			boolean isPKClassName = false;
			String searchClassName = className;
			String searchFieldName = fieldName;

			// translate the class name & field names to corresponding 
			// ejb name is abstract bean equivalents if necessary
			if (isPCClass)
			{
				searchFieldName = nameMapper.
					getEjbFieldForPersistenceField(className, fieldName);
				searchClassName = getEjbName(className);
			}
			else	// check if it is a pk class without a user defined key class
			{
				String ejbName = 
					nameMapper.getEjbNameForPersistenceKeyClass(className);

				switch (getPersistenceKeyClassType(className))
				{
					// find the field we need in the corresponding 
					// abstract bean (translated below from ejbName)
					case NameMapper.PRIMARY_KEY_FIELD:
						testClass = ejbName;
						searchClassName = ejbName;
						isPKClassName = true;
						break;
					// find the field we need by called updateFieldWrapper
					// below which handles the generated field for the 
					// unknown key class - need to use the
					// persistence-capable class name and flag to call that
					// code, so we configure it here
					case NameMapper.UNKNOWN_KEY_CLASS:
						testClass = nameMapper.
							getPersistenceClassForEjbName(ejbName);
						isPCClass = true;
						isPKClassName = true;
						break;
				}
			}

			if (nameMapper.isEjbName(searchClassName))
			{
				searchClassName = nameMapper.
					getAbstractBeanClassForEjbName(searchClassName);
			}

			returnObject = super.getField(searchClassName, searchFieldName);

			if (returnObject == null)	// try getting it from the descriptor
				returnObject = getFieldWrapper(testClass, searchFieldName);
			else if (returnObject instanceof Field)		// wrap it
				returnObject = new MemberWrapper((Field)returnObject);

			if (isPCClass)
			{
				returnObject = updateFieldWrapper(
					(MemberWrapper)returnObject, testClass, fieldName);
			}
			// when asking for these fields as part of the 
			// persistence-capable is key class, we need to represent the 
			// public modifier which will be generated in the inner class
			if (isPKClassName && (returnObject instanceof MemberWrapper))
				((MemberWrapper)returnObject)._modifiers = Modifier.PUBLIC;
			
		}

		return returnObject;
	}
\end{code}
\end{lstlisting}
	

\newpage
\subsection{Fifth Method: getFieldType} \\ \\
	/** Returns the field type for the specified fieldName in the class
	 * with the specified className.  This method is overrides the one in 
	 * Model in order to do special handling for non-collection relationship 
	 * fields.  If it's a generated relationship that case, the returned 
	 * MemberWrapper from getField contains a type of the abstract bean and 
	 * it's impossible to convert that into the persistence capable class name, so here 
	 * that case is detected, and if found, the ejb name is extracted and 
	 * used to find the corresponding persistence capable class.  For a 
	 * relationship which is of type of the local interface, we do the 
	 * conversion from local interface to persistence-capable class.  In the 
	 * case of a collection relationship (generated or not), the superclass' 
	 * implementation which provides the java type is sufficient.
	 * @param className the fully qualified name of the class which contains
	 * the field to be checked
	 * @param fieldName the name of the field to be checked
	 * @return the field type for the specified fieldName
	 */
	public String getFieldType (String className, String fieldName)
	{
		String returnType = super.getFieldType(className, fieldName);

		if (!isCollection(returnType) && isPCClassName(className))
		{
			NameMapper nameMapper = getNameMapper();
			String ejbName = 
				nameMapper.getEjbNameForPersistenceClass(className);
			String ejbField = 
				nameMapper.getEjbFieldForPersistenceField(className, fieldName);

			if (nameMapper.isGeneratedEjbRelationship(ejbName, ejbField))
			{
				String[] inverse = 
					nameMapper.getEjbFieldForGeneratedField(ejbName, ejbField);
            
				returnType = nameMapper.
					getPersistenceClassForEjbName(inverse[0]);
			}

			if (nameMapper.isLocalInterface(returnType))
			{
				returnType = nameMapper.getPersistenceClassForLocalInterface(
					className, fieldName, returnType);
			}
		}

		return returnType;
	}



\newpage
\subsection{Sixth Method: getFieldWrapper} \\ \\non c'è commento.. asd

\begin{lstlisting}

private MemberWrapper getFieldWrapper (String className, String fieldName)
	{
		EjbCMPEntityDescriptor descriptor = getCMPDescriptor(className);
		MemberWrapper returnObject = null;

		if (descriptor != null)
		{
			PersistenceDescriptor persistenceDescriptor =
				descriptor.getPersistenceDescriptor();

			if (persistenceDescriptor != null)
			{
				Class fieldType = null;

				try
				{
					fieldType = persistenceDescriptor.getTypeFor(fieldName);
				}
				catch (RuntimeException e)
				{
					// fieldType will be null - there is no such field
				}

				returnObject = ((fieldType == null) ? null :
					new MemberWrapper(fieldName, fieldType, 
					Modifier.PRIVATE, (Class)getClass(className)));
			}
		}

		return returnObject;
	}

\end{lstlisting}

\section{Functional Roles}
\subsection{DeploymentDescriptorModel class}
The main function of this class is to augment the java metadata for a non-existent persistence-capable java/class file using the deployment descriptor. It is primarily used at ejbc time, though it could be used at any time as long as sufficient mapping and deployment descriptor information is available.

\subsection{MemberWrapper class}
This class provides functionalities used to wrap an element such as a constructor or a method into an object, so that the element can be used as an object.

\subsection{getConstructor method}
The role of this method is to provide a wrapped constructor element for the class identified by the className parameter, this works only if the class isn't a persistence-capable key class that corresponds to a bean with unknown primary key, in this case a dummy constructor is returned. For a more detailedm description and implementation see the javadoc related to this method.

\subsection{getMethod method}
This method, similarly to the previous one, is used to get a wrapped constructor element for the class identified by the className parameter. Also in this case if the class is a persistence capable key class which corresponds to a bean with unknown primary key which or a primary key field, and the method identified is a equals or hashCode name a dummy method is returned. A dummy method is returned also in the case where the class is persistence capable and the method is a read or writed object method. This method will never returns inherited methods, for this purpose is used the getInheritedMethod method.

\subsection{getFields method}
The role of this method is to return a list of names of all the declared field elements in the class identified by the className parameter. If the class represents a persistence-capable class, the list of field names from the corresponding ejb is returned.

\subsection{getField method}
The role of this method is similar to the previous one but instead this one return only one wrapped field element identified by the fieldName parameter.

\subsection{getFieldType method}
This method works as the previous one but instead this returns the field type (as a string) of a field element identified by the fieldName parameter in the specified class.

\subsection{getFieldWrapper method}
The role of this method is to provide a functionality to get a field wrapper given a field's name and type.

\section{Issues found}
In this section we will list all the issues found by applying the checklist provided to the methods and classes we were assigned.

\subsection{DeploymentDescriptorModel class issues}
\begin{itemize}
	\item In several lines of the class provided the indentation is done by tab and not by space, although it is equal to 4 spaces in every line.
	\item The bracing style chosen is Allman style and is consistent in every line except lines 111, 112, 918 to 922 and 927, where fact curly braces and statement between them is written in one line only. 
	\item No line exceed the 80 characters limit except line 95 and line 276, although they not exceed the 120 characters limit.
	\item The DeploymentDescriptorModel.java file contains the public class DeploymentDescriptorModel that is the first class declared, and the private class MemberWrapper.
	\item MemberWrapper class is not documented properly.
	\item The javadoc is missing for the following methods: getEjbName, getPersistenceKeyClassType, getFieldWrapper, updateFieldWrapper.
	\item Variables are declared in the wrong order, first are declared non-static instance variables and then static class variables.
	\item Methods are grouped properly by functionalities.
	\item getField and updateFielWrapper methods are too long, they should split in two or more auxiliary methods.
\end{itemize}

\subsection{Member wrapper class issues}
\begin{itemize}
	\item The class is not documented properly
	\item Lines 902 to 907 there is a large use of ? : operator instead of if-else statements, this makes the code much more difficult to read and understand.
\end{itemize}

\subsection{getConstructor method issues}
\begin{itemize}
	\item Line 285: if statement with only one statement to execute without curly braces.
	\item Lines are wrapped properly
	\item The method is documented properly
	\item There isn't commented out code.
	\item All declarations and initializations are done in the correct manner.
	\item All methods are called correctly.
	\item Method doesn't make use of arrays.
	\item Line: 276: two strings are compared using == operator instead of equals method.
	\item There are no computations. All assignments are done correctly and there are no implicit casts.
	\item There are switch statements or loops.
	\item There is no use of files.
\end{itemize}

\subsection{getMethod method issues}
\begin{itemize}
	\item Line 349: if statement with only one statement to execute without curly braces.
	\item Lines are wrapped properly
	\item The method is documented properly
	\item There isn't commented out code.
	\item All declarations and initializations are done in the correct manner.
	\item All methods are called correctly.
	\item Method doesn't make use of arrays.
	\item Line 328 and 329: two string are compared using  == operator instead of equals method.
	\item There are no computations. All assignments are done correctly and there are no implicit casts.
	\item There are switch statements or loop.
	\item There is no use of files.
\end{itemize}

\subsection{getFields method issues}
\begin{itemize}
	\item Line 403: while loop with only one statement to execute without curly braces.
	\item Line 435: if statement with only one statement to execute without curly braces.
	\item Lines are wrapped properly
	\item The method is documented properly
	\item There isn't commented out code.
	\item All declarations and initializations are done in the correct manner.
	\item All methods are called correctly.
	\item Arrays are implemented correctly, there shouldn't be any indexes error since an iterator is used.
	\item Objects are compared correctly.
	\item There are no computations. All assignments are done correctly and there are no implicit casts.
	\item All cases are addressed with break statement.
	\item There is no use of files.
\end{itemize}

\subsection{getField method issues}
\begin{itemize}
	\item This method is too long, it should be split in two or more auxiliary methods.
	\item Line 539-541: if-else statement with only one statement to execute without curly braces.
	\item Line 552, if statement with only one statement to execute without curly braces.
	\item Lines are wrapped properly
	\item The method is documented properly
	\item There isn't commented out code.
	\item All declarations and initializations are done in the correct manner.
	\item All methods are called correctly.
	\item Objects are compared correctly.
	\item There are no computations. All assignments are done correctly and there are no implicit casts.
	\item All cases are addressed with break statement.
	\item There is no use of files.
\end{itemize}

\subsection{getFieldType method issues}
\begin{itemize}
	\item Lines are wrapped properly
	\item The method is documented properly
	\item There isn't commented out code.
	\item All declarations and initializations are done in the correct manner.
	\item All methods are called correctly.
	\item Objects are compared correctly.
	\item There are no computations. All assignments are done correctly and there are no implicit casts.
	\item There are no switch case statements
	\item There is no use of files.
\end{itemize}

\subsection{getFieldWrapper method issues}
\begin{itemize}
	\item Lines are wrapped properly
	\item The method is documented properly
	\item There isn't commented out code.
	\item All declarations and initializations are done in the correct manner.
	\item All methods are called correctly.
	\item Objects are compared correctly.
	\item There are no computations. All assignments are done correctly and there are no implicit casts.
	\item Line 794 to 801, there is a try-catch block but no actions are taken to handle the exception thrown.
	\item All cases are addressed with break statement.
\end{itemize}

\section{Other Problems}

\subsection{getConstructor problems}
\begin{itemize}
	\item Line 276: The comparison between two strings using == operator may lead to a runtime error.
\end{itemize}

\subsection{getMethod problems}
\begin{itemize}
	\item Line 328-329: The comparison between two strings using == operator may lead to a runtime error.
\end{itemize}

\subsection{getFields problems}
We haven't highlighted any problems or bugs for this method, it should work as planned.

\subsection{getField problems}
We haven't highlighted any problems or bugs for this method, it should work as planned.

\subsection{getFieldType problems}
We haven't highlighted any problems or bugs for this method, it should work as planned.

\subsection{getFieldWrapper problems}
We haven't highlighted any problems or bugs for this method, it should work as planned.

\section{APPENDIX}

\subsection{Used tools}
To create the Code Inspection document we used the following tools:
\begin{itemize}
	\item \textbf{MikTex} \\ Distribution of the typesetting system LaTeX \\ \url{http://www.miktex.org/download } 
	\item \textbf{TexStudio}\\ OpenSource cross-platform LaTeX editor we used to write the code inspection document. \\ \url{http://texstudio.sourceforge.net  } 
	\item \textbf{GitHub desktop}\\ Desktop application of the web-based Git repository hosting service. Used to collaborate in the team and to have a track of the changes.  \\ \url{https://desktop.github.com/ } 
\end{itemize}

\subsection{Reference}
\begin{itemize}
	\item Glassfish javadoc: http://glassfish.pompel.me/
	\item Code inspection assignment and checklist: Assignment 3 - code inspection.pdf
	\item Code to be inspected: DeploymentDescriptorModel.java
\end{itemize}

\subsection{Hours of work}
This is the time we spent inspecting the code and redacting the document.
\begin{itemize}
	\item {Belotti Nicola} \textasciitilde 11 hours
	\item {Chioso Emanuele} \textasciitilde 11 hours
	\item {Colombo andrea} \textasciitilde 11 hours
\end{itemize}

\end{document}


\begin{comment} ROBE DA FARE DAL PDF DELLA PROF. ******* TO DO ******
Suggested structure for the inspection document
Front page: Include at least the project title, the version of the document, your
names and the release date
Table of content: Include the table of content of your document
   
Classes that were assigned to the group: <state the namespace pattern and name of the classes that were assigned to you>

Functional role of assigned set of classes: <elaborate on the functional role you have identified for the class cluster that was assigned to you, also, elaborate on how you managed to understand this role and provide the necessary evidence, e.g., javadoc, diagrams, etc.>

List of issues found by applying the checklist: <report the classes/code fragments that do not fulfill some points in the check list. Explain which point is not fulfilled and why>.

Any other problem you have highlighted: <list here all the parts of code that you think create or may create a bug and explain why>.
\end{comment}





