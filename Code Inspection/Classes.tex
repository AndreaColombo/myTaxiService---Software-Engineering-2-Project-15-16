\section{Assigned Methods}
\subsection{First Method: getConstructor} \\ \\

	/** Returns a wrapped constructor element for the specified argument types 
	 * in the class with the specified name.  If the specified class name is 
	 * a persistence-capable key class name which corresponds to a bean 
	 * with an unknown primary key class a dummy constructor will also be 
	 * returned.  Types are specified as type names for primitive type 
	 * such as int, float or as fully qualified class names.
	 * @param className the name of the class which contains the constructor 
	 * to be checked
	 * @param argTypeNames the fully qualified names of the argument types
	 * @return the constructor element
	 * @see #getClass
	 */
	 
\begin{lstlisting}	public Object getConstructor (final String className, String[] argTypeNames)
	{
		Object returnObject = null;

		if ((NameMapper.PRIMARY_KEY_FIELD == 
			getPersistenceKeyClassType(className)) && 
			Arrays.equals(argTypeNames, NO_ARGS))
		{
			returnObject = new MemberWrapper(className, null, Modifier.PUBLIC, 
				(Class)getClass(className));
		}

		if (returnObject == null)
		{
			returnObject = super.getConstructor(className, argTypeNames);

			if (returnObject instanceof Constructor)		// wrap it
				returnObject = new MemberWrapper((Constructor)returnObject);
		}

		return returnObject;
	}
	
\end{lstlisting}
	
	
\newpage
\subsection{Second Method: getMethod} \\ \\


/** Returns a wrapped method element for the specified method name and \\
	 * argument types in the class with the specified name.  If the \\
	 * specified className represents a persistence-capable class and \\
	 * the requested methodName is readObject or writeObject, a dummy \\
	 * method will be returned.  Similarly, if the specified class name is 
	 * a persistence-capable key class name which corresponds to a bean 
	 * with an unknown primary key class or a primary key field (in both 
	 * cases there is no user defined primary key class) and the requested 
	 * method is equals or hashCode, a dummy method will also be returned.  
	 * Types are specified as type names for primitive type such as int, 
	 * float or as fully qualified class names.  Note, the method does not 
	 * return inherited methods.
	 * @param className the name of the class which contains the method 
	 * to be checked
	 * @param methodName the name of the method to be checked
	 * @param argTypeNames the fully qualified names of the argument types
	 * @return the method element
	 * @see #getClass
	 */
\begin{lstlisting}
	
	public Object getMethod (final String className, final String methodName,
		String[] argTypeNames)
	{
		int keyClassType = getPersistenceKeyClassType(className);
		Object returnObject = null;

		if (isPCClassName(className))
		{
			if ((methodName.equals("readObject") && 	// NOI18N
                    Arrays.equals(argTypeNames, getReadObjectArgs())) ||
				(methodName.equals("writeObject") && 	// NOI18N
                        Arrays.equals(argTypeNames, getWriteObjectArgs())))
			{
				returnObject = new MemberWrapper(methodName, 
					Void.TYPE, Modifier.PRIVATE, (Class)getClass(className));
			}
		}
		if ((NameMapper.UNKNOWN_KEY_CLASS == keyClassType) || 
			(NameMapper.PRIMARY_KEY_FIELD == keyClassType))
		{
			if (methodName.equals("equals") && 	// NOI18N
                    Arrays.equals(argTypeNames, getEqualsArgs()))
			{
				returnObject = new MemberWrapper(methodName, 
					Boolean.TYPE, Modifier.PUBLIC, (Class)getClass(className));
			}
			else if (methodName.equals("hashCode") && 	// NOI18N
                    Arrays.equals(argTypeNames, NO_ARGS))
			{
				returnObject = new MemberWrapper(methodName, 
					Integer.TYPE, Modifier.PUBLIC, (Class)getClass(className));
			}
		}

		if (returnObject == null)
		{
			returnObject = super.getMethod(className, methodName, argTypeNames);

			if (returnObject instanceof Method)		// wrap it
				returnObject = new MemberWrapper((Method)returnObject);
		}

		return returnObject;
	}	
\end{lstlisting}

\newpage
\subsection{Third Method: getFields} \\ \\

/** Returns a list of names of all the declared field elements in the 
	 * class with the specified name.  If the specified className represents 
	 * a persistence-capable class, the list of field names from the  
	 * corresponding ejb is returned (even if there is a Class object 
	 * available for the persistence-capable).
	 * @param className the fully qualified name of the class to be checked 
	 * @return the names of the field elements for the specified class
	 */
\begin{lstlisting}
	
	public List getFields (final String className)
	{
		final EjbCMPEntityDescriptor descriptor = getCMPDescriptor(className);
		String testClass = className;

		if (descriptor != null)		// need to get names of ejb fields
		{
			Iterator iterator = descriptor.getFieldDescriptors().iterator();
			List returnList = new ArrayList();

			while (iterator.hasNext())
				returnList.add(((FieldDescriptor)iterator.next()).getName());

			return returnList;
		}
		else
		{
			NameMapper nameMapper = getNameMapper();
			String ejbName = 
				nameMapper.getEjbNameForPersistenceKeyClass(className);

			switch (getPersistenceKeyClassType(className))
			{
				// find the field names we need in the corresponding 
				// ejb key class
				case NameMapper.USER_DEFINED_KEY_CLASS:
					testClass = nameMapper.getKeyClassForEjbName(ejbName);
					break;
				// find the field name we need in the abstract bean 
				case NameMapper.PRIMARY_KEY_FIELD:
					return Arrays.asList(new String[]{
						getCMPDescriptor(ejbName).
						getPrimaryKeyFieldDesc().getName()});
				// find the field name we need in the persistence capable 
				case NameMapper.UNKNOWN_KEY_CLASS:
					String pcClassName = 
						nameMapper.getPersistenceClassForEjbName(ejbName);
					PersistenceFieldElement[] fields = 
						getPersistenceClass(pcClassName).getFields();
					int i, count = ((fields != null) ? fields.length : 0);

					for (i = 0; i < count; i++)
					{
						PersistenceFieldElement pfe = fields[i];

						if (pfe.isKey())
							return Arrays.asList(new String[]{pfe.getName()});
					}
					break;
			}
		}

		return super.getFields(testClass);
	}
\end{lstlisting}


\newpage
\subsection{Fourth Method: getField} \\ \\
	/** Returns a wrapped field element for the specified fieldName in the 
	 * class with the specified className.  If the specified className 
	 * represents a persistence-capable class, a field representing the 
	 * field in the abstract bean class for the corresponding ejb is always 
	 * returned (even if there is a Field object available for the 
	 * persistence-capable).  If there is an ejb name and an abstract bean
	 * class with the same name, the abstract bean class which is associated
	 * with the ejb will be used, not the abstract bean class which  
	 * corresponds to the supplied name (directly).
	 * @param className the fully qualified name of the class which contains
	 * the field to be checked
	 * @param fieldName the name of the field to be checked
	 * @return the wrapped field element for the specified fieldName
	 */
	 
\begins{lstlisting}
\begin{code}
public Object getField (final String className, String fieldName)
	{
		String testClass = className;
		Object returnObject = null;

		if (className != null)
		{
			NameMapper nameMapper = getNameMapper();
			boolean isPCClass = isPCClassName(className);
			boolean isPKClassName = false;
			String searchClassName = className;
			String searchFieldName = fieldName;

			// translate the class name & field names to corresponding 
			// ejb name is abstract bean equivalents if necessary
			if (isPCClass)
			{
				searchFieldName = nameMapper.
					getEjbFieldForPersistenceField(className, fieldName);
				searchClassName = getEjbName(className);
			}
			else	// check if it is a pk class without a user defined key class
			{
				String ejbName = 
					nameMapper.getEjbNameForPersistenceKeyClass(className);

				switch (getPersistenceKeyClassType(className))
				{
					// find the field we need in the corresponding 
					// abstract bean (translated below from ejbName)
					case NameMapper.PRIMARY_KEY_FIELD:
						testClass = ejbName;
						searchClassName = ejbName;
						isPKClassName = true;
						break;
					// find the field we need by called updateFieldWrapper
					// below which handles the generated field for the 
					// unknown key class - need to use the
					// persistence-capable class name and flag to call that
					// code, so we configure it here
					case NameMapper.UNKNOWN_KEY_CLASS:
						testClass = nameMapper.
							getPersistenceClassForEjbName(ejbName);
						isPCClass = true;
						isPKClassName = true;
						break;
				}
			}

			if (nameMapper.isEjbName(searchClassName))
			{
				searchClassName = nameMapper.
					getAbstractBeanClassForEjbName(searchClassName);
			}

			returnObject = super.getField(searchClassName, searchFieldName);

			if (returnObject == null)	// try getting it from the descriptor
				returnObject = getFieldWrapper(testClass, searchFieldName);
			else if (returnObject instanceof Field)		// wrap it
				returnObject = new MemberWrapper((Field)returnObject);

			if (isPCClass)
			{
				returnObject = updateFieldWrapper(
					(MemberWrapper)returnObject, testClass, fieldName);
			}
			// when asking for these fields as part of the 
			// persistence-capable is key class, we need to represent the 
			// public modifier which will be generated in the inner class
			if (isPKClassName && (returnObject instanceof MemberWrapper))
				((MemberWrapper)returnObject)._modifiers = Modifier.PUBLIC;
			
		}

		return returnObject;
	}
\end{code}
\end{lstlisting}
	

\newpage
\subsection{Fifth Method: getFieldType} \\ \\
	/** Returns the field type for the specified fieldName in the class
	 * with the specified className.  This method is overrides the one in 
	 * Model in order to do special handling for non-collection relationship 
	 * fields.  If it's a generated relationship that case, the returned 
	 * MemberWrapper from getField contains a type of the abstract bean and 
	 * it's impossible to convert that into the persistence capable class name, so here 
	 * that case is detected, and if found, the ejb name is extracted and 
	 * used to find the corresponding persistence capable class.  For a 
	 * relationship which is of type of the local interface, we do the 
	 * conversion from local interface to persistence-capable class.  In the 
	 * case of a collection relationship (generated or not), the superclass' 
	 * implementation which provides the java type is sufficient.
	 * @param className the fully qualified name of the class which contains
	 * the field to be checked
	 * @param fieldName the name of the field to be checked
	 * @return the field type for the specified fieldName
	 */
	public String getFieldType (String className, String fieldName)
	{
		String returnType = super.getFieldType(className, fieldName);

		if (!isCollection(returnType) && isPCClassName(className))
		{
			NameMapper nameMapper = getNameMapper();
			String ejbName = 
				nameMapper.getEjbNameForPersistenceClass(className);
			String ejbField = 
				nameMapper.getEjbFieldForPersistenceField(className, fieldName);

			if (nameMapper.isGeneratedEjbRelationship(ejbName, ejbField))
			{
				String[] inverse = 
					nameMapper.getEjbFieldForGeneratedField(ejbName, ejbField);
            
				returnType = nameMapper.
					getPersistenceClassForEjbName(inverse[0]);
			}

			if (nameMapper.isLocalInterface(returnType))
			{
				returnType = nameMapper.getPersistenceClassForLocalInterface(
					className, fieldName, returnType);
			}
		}

		return returnType;
	}



\newpage
\subsection{Sixth Method: getFieldWrapper} \\ \\non c'è commento.. asd

\begin{lstlisting}

private MemberWrapper getFieldWrapper (String className, String fieldName)
	{
		EjbCMPEntityDescriptor descriptor = getCMPDescriptor(className);
		MemberWrapper returnObject = null;

		if (descriptor != null)
		{
			PersistenceDescriptor persistenceDescriptor =
				descriptor.getPersistenceDescriptor();

			if (persistenceDescriptor != null)
			{
				Class fieldType = null;

				try
				{
					fieldType = persistenceDescriptor.getTypeFor(fieldName);
				}
				catch (RuntimeException e)
				{
					// fieldType will be null - there is no such field
				}

				returnObject = ((fieldType == null) ? null :
					new MemberWrapper(fieldName, fieldType, 
					Modifier.PRIVATE, (Class)getClass(className)));
			}
		}

		return returnObject;
	}

\end{lstlisting}